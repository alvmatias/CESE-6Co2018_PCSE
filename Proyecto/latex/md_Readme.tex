\subsection*{Compilación}

Recordar compilar con \char`\"{}make\char`\"{} desde adentro esta misma carpeta.

De manera similar al \char`\"{}\+Makefile.\+mine\char`\"{} del C\+I\+AA Firmware tenemos el archivo \char`\"{}project.\+mk\char`\"{} que permite seleccionar el proyecto y placa objetivo a compilar.

\subsection*{Utilización}

Este ejemplo escribe un archivo de texto en una tarjeta SD card mediante S\+PI.

Funciona sobre la E\+D\+U-\/\+C\+I\+A\+A-\/\+N\+XP. El conexionado utilizando el lector de SD es\+:

S\+D/\+Micro\+SD Card Reader --$>$ E\+D\+U-\/\+C\+I\+A\+A-\/\+N\+XP \begin{DoxyVerb}                 + --> +3.3V
                CS --> GPIO0
                DI --> SPI_MOSI
               CLK --> SPI_SCK
                DO --> SPI_MISO
                 G --> GND
\end{DoxyVerb}


Utiliza el modulo Fat\+Fs (\href{http://elm-chan.org/fsw/ff/00index_e.html}{\tt http\+://elm-\/chan.\+org/fsw/ff/00index\+\_\+e.\+html}) y las funciones de la A\+PI para S\+SP de L\+P\+C\+Open (\href{https://www.lpcware.com/lpcopen}{\tt https\+://www.\+lpcware.\+com/lpcopen}).

Se debe conectar la SD antes de correr el ejemplo, si prende el L\+E\+DG significa que pudo grabar el archivo correctamente en la tarjeta SD.

No se encuentran implementados aun\+: File\+Read, timestamp, debug messages, etc. 